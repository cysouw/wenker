\documentclass[11pt]{article}

% fonts

\usepackage[charter,expert]{mathdesign}
\usepackage{fontspec}
\defaultfontfeatures{Mapping=tex-text}
\setmainfont[Scale=0.95]{CharisSIL}
\setmonofont[Scale=0.8]{Monaco}

% footnote format

\usepackage[hang,ragged]{footmisc}

% pdf with functioning links

\usepackage{ifpdf}
\ifpdf 
    \usepackage[pdftex]{graphicx}   % to include graphics
    \pdfcompresslevel=9 
    \usepackage[pdftex,     % sets up hyperref to use pdftex driver
            plainpages=false,   % allows page i and 1 to exist in the same document
            breaklinks=true,    % link texts can be broken at the end of line
            colorlinks=true,
            hidelinks, % do not show the links with colors or underlining
            pdfauthor={$author-meta$},
            pdftitle={$title-meta$}
           ]{hyperref} 
    \usepackage{thumbpdf}
\else 
    \usepackage{graphicx}       % to include graphics
    \usepackage[hidelinks]{hyperref}       % to simplify the use of \href
\fi 

% bibliography

\usepackage[backend=biber,
            bibstyle=biblatex-sp-unified,
            citestyle=sp-authoryear-comp
            ]{biblatex}
\addbibresource{/Users/cysouw/Documents/Linguistics/michaellibrary.bib}



\title{Analysing translational survey data through multialignment}
\author{Michael Cysouw \and Jürg Fleischer}
\date{\today}

\begin{document}
  
\maketitle
  
\begin{abstract}
In the analysis of (dialectal) survey data there is a long path of interpretation from the data that is collected to the eventual interpretation. In most past (and current) research there is no paper-trail of all the large and small decisions being taken in the processing of the data. This paper describes as series of methods to document the processing of translational survey data, i.e. data that consists of translational equivalents. As an example we will process 2500 translations of a single sentence from the original Wenker data, transliterated from the original questionnaire from the 19th century. In addition to the well-known geographic distribution of sounds, we will show that it is just as well possible to extract syntactic and lexical variables from this data.
\end{abstract}

\section{Multialigning translations}

There are many ways in which comparable data can be collected to compare different language variants. Possibly the most traditional kind of comparable data (and also often critcized, REF?) are translational equivalent utterances. Language consultants are simply asked to produce the closest possible translation of a given utterance in their language. It is this kind of data that will be the focus of the paper, though the techniques proposed have a much wider application. We will analyse Wenker sentence 9: `\textit{Ich bin bei der Frau gewesen und ich habe es ihr gesagt, und sie sagte, sie wolle es auch ihrer Tochter sagen.}' (I have been at the women and I have told it to her, and she said that she would tell it to her daughter). We have transliterated about 2500 translations from the original Wenker questionnaire, extended with translations from Austria, Switzerland, the Netherlands, Belgium and various german-speaking linguistic enclaves. This data was used at the start of the 20th century to produce the infamous dialectmaps of

%\printbibliography
\end{document}  
